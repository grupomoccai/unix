\input{macros.tex}

\title{Introducción a UNIX: Trabajo Práctico 5}
\author{Daniel Millán \& Nicolás Muzi}
\date{\textit{Facultad de Ciencias Aplicadas a la Industria, UNCuyo}\\ San Rafael 5600, Argentina\\Mayo -- Junio de 2019}    

\begin{document}

\begin{tabular}{ccc}
\includegraphics[width=0.1\textwidth]{pingu.jpg} &
\begin{minipage}[t!]{0.75\textwidth}
\maketitle
\end{minipage} &
\includegraphics[width=0.1\textwidth]{logouncuyo.jpg}\\
\end{tabular}
\hrule
\vspace{5mm}

%%%%%%%%%%%%%%%%%%%%%%%%%%%%%%%%%%%%%%%%%%%%%%%%%%%%
%%%%%%%%%%%%%%%%%%%%%%%%%%%%%%%%%%%%%%%%%%%%%%%%%%%%

\begin{ejercicio}
Cree un \script~ ``{\it aproxima}'' el cual genera un número aleatorio entre 1 y 100 (ver TP2, Ejercicio 4.1), e indica pistas al usuario hasta que adivina el número. El \script~debe dar pistas del estilo ``Lo siento tu intento es demasiado alto'', o ``Intenta nuevamente con un numero menor'' (no utilice acentos). 
\end{ejercicio}


%%%%%%%%%%%%%%%%%%%%%%%%%%%%%%%%%%%%%%%%%%%%%%%%%%%%

\begin{ejercicio}
\begin{enumerate}
\item Escriba un \shell~\script~que cambie la extensión de todos los archivos \ttt{.txt} por la extensión \ttt{.text}. Podría serle de ayuda la orden \ttt{basename}.
\item Modifique el \script~anterior (cree uno nuevo) para que en caso de ser indicado cambie de una extensión dada a otra cualquiera también definida por el usuario.
\end{enumerate}
\end{ejercicio}


%%%%%%%%%%%%%%%%%%%%%%%%%%%%%%%%%%%%%%%%%%%%%%%%%%%%
\begin{ejercicio}
\begin{enumerate}
 \item Cree un \shell~\script~``{\it pidof}'' el cual dado un nombre como parámetro devuelve el PID(s) del proceso con ese nombre.
 \item Agregue una función \ttt{usage()} al \shell~\script~``{\it pidof}'' creado anteriormente la cual imprime información sobre el modo de uso. La función debe ser llamada en caso de pasar un número erróneo de parámetros.
 \end{enumerate}
\end{ejercicio}


%%%%%%%%%%%%%%%%%%%%%%%%%%%%%%%%%%%%%%%%%%%%%%%%%%%%
\begin{ejercicio}
\begin{enumerate}
\item ¿Qué sucede si la función \ttt{findfile} dada en las notas es agregada en su \ttt{.bashrc}?
\item Cree un \script~``{\it swap}'' tal que dados dos archivos comunes de entrada intercambie los contenidos entre ellos. ¿Es conveniente utilizar una función \ttt{swapping()}?
\item ¿Dónde se supone que debe colocar las siguientes funciones y qué hacen?
\begin{verbatim}
ls() { /bin/ls -sbF "$@";} 
ll() { ls -al "$@";}
\end{verbatim}

\end{enumerate}
\end{ejercicio}

%%%%%%%%%%%%%%%%%%%%%%%%%%%%%%%%%%%%%%%%%%%%%%%%%%%%
\pagebreak
\begin{ejercicio}
\begin{enumerate}

\item Cree el siguiente \script~dado en clase
\begin{verbatim}
#!/bin/sh
for f in *.txt
do
    echo ordenando archivo $f
    cat $f | sort > $f.sorted
    echo archivo ordenado ha sido redirecconado a $f.sorted
done
\end{verbatim}

Modifique el \script~(puede hacerlo en \ttt{bash}) tal que guarde el archivo con extensión \ttt{.txts} en lugar de \ttt{.txt.sorted}.

\item Ahora cree el \shell~\script~``{\it waitinput}'' cuyo contenido se indica a continuación (dado en las notas):
\begin{verbatim}
#!/bin/sh
while [ ! -s input.txt ] do
    echo waiting ...
    sleep 5
done
echo input.txt is ready
\end{verbatim}

Pruebe si funciona, en caso de que no funcione ¿Qué error tiene, puede corregirlo?
\end{enumerate}
\end{ejercicio}


%%%%%%%%%%%%%%%%%%%%%%%%%%%%%%%%%%%%%%%%%%%%%%%%%%%%
\begin{ejercicio}
\begin{enumerate}

\item Realice un \shell~\script~que permita crear un conjunto $N$ de directorios, numerados que comiencen con el mismo ``nombre''. Debe chequear que dicho número $N$ no exceda 1000, y que el largo de la cadena de caracteres del ``nombre'' no sea mayor a 8. 
\item Crear una función \ttt{usage()} que brinde información sobre como se debe emplear este \shell~\script~en caso de error.

\item Cree un script que ``renombre'' los directorios creados en el apartado anterior.

\end{enumerate}
\end{ejercicio}

%%%%%%%%%%%%%%%%%%%%%%%%%%%%%%%%%%%%%%%%%%%%%%%%%%%%
\begin{ejercicio}
\begin{enumerate}
\item ¿Puede explicar cada línea del siguiente \shell~\script? ¿Qué hace?
\begin{verbatim}
#!/bin/sh
while [ $# -gt 0 ] 
do
    echo $1
    shift 
done
\end{verbatim}

\item ¿Qué diferencia existe con el siguiente \script?
\begin{verbatim}
#!/bin/sh
until [ $# -le 0 ] 
do
    echo $1
    shift 
done
\end{verbatim}
\end{enumerate}
\end{ejercicio}

%%%%%%%%%%%%%%%%%%%%%%%%%%%%%%%%%%%%%%%%%%%%%%%%%%%%
\begin{ejercicio}
Dado un conjunto no definido pero finito de archivos comunes del tipo \ttt{mod\_rotu\_X.vtk} se desea contar con un programa que les permita cambiar el nombre a \ttt{modelo\_rotura\_YYYY.itk}. ¿Cómo haría esto?. Realice el \shell~\script~que permita resolver este problema. (Nota: X representa un número entre 0 y 9999, e YYYY un número del tipo 0000 a 9999).
\end{ejercicio}
\vspace*{0.5cm}

%%%%%%%%%%%%%%%%%%%%%%%%%%%%%%%%%%%%%%%%%%%%%%%%%%%%
\begin{ejercicio}
En una terminal pruebe:
\begin{verbatim}
$ wget https://grupomoccai.github.io/unix/practicas_archivos/aneuweb/...
ICA_centerlines/C0001_centerlines.vtk

$ wget https://grupomoccai.github.io/unix/practicas_archivos/aneuweb/...
ICA_automatednecks/C0001_automatedneck_new.vtk

$ wget https://grupomoccai.github.io/unix/practicas_archivos/aneuweb/...
ICA_models/C0001_model.vtk

\end{verbatim}
\textit{Nota: Los puntos suspensivos sólo se usaron por falta de espacio. Ingrese cada ruta en una sola línea, sin los puntos.}\\

Cree un \script~que realice la descarga de estos archivos para los casos del 05-14 y del 31-40. Este \script~debe:

\begin{enumerate}
\item Crear un directorio para cada caso, cuyo formato debe ser \ttt{CXXXX}.
\item Renombrar cada archivo como \ttt{CXXXX\_clines.vtk}, \ttt{CXXXX\_neck.vtk} y \ttt{CXXXX\_geomodel.vtk}, respectivamente.
\end{enumerate}
\end{ejercicio}

\begin{ejercicio}
Cree un script que le permita ejecutar el programa BioAneuIsolation para una lista de casos seleccionados. Ejecútelo para los casos descargados en el ejercicio anterior, guardando la salida \ttt{CXXXX\_output.vtk} en el directorio correspondiente a cada caso \ttt{CXXXX}. 
\end{ejercicio}

\subsection*{Entrega}
Se entregarán los ejercicios 3, 5, 6.1, 6.2, 7 y 10. 
%%%%

\end{document}  
